\section{Bürokratie}

\subsection{Allgemein}
Bei der Bürokratie soll den Ersties gezeigt werden wie es später ist auf verschieden Ämtern zu warten nur um dann zu erfahren das sie das richtige Formular nicht ausgefüllt haben und so das warten noch einmal von Vorne beginnen darf. Sollte alles erfolgreich absolviert sein und man steht vor dem Beamten, so kann es immer noch passieren das dieser aufgrund von akutem Koffeinmangel ein schläft.

\subsection{Aufbau}
Der Raum soll möglichst kompliziert aufgebaut werden, so das bereits das warten zum Verzweifeln ist. Ein Labyrinth aus Stühlen, gefolgt von Klebeband auf dem Boden, welches keinesfalls übertreten werden darf, könnte eine Idee sein. Schilder an den Wänden, mit Hinweisen wie man mit schlafenden Beamten umgeht, dürfen natürlich auch nicht fehlen. Zur selben Zeit sollten nicht alle Ämter besetzt sein. Es werden zur selben Zeit in der Regel 4 Engel benötigt, jedoch sollten 6 Engeln mit den Aufgaben vertraut sein, damit Pausen möglich sind. Auch wenn die Engel nicht zu 100\% mit den Aufgaben vertraut sein müssen, eigentlich auch nicht sollten, so sollten der grobe Ablauf trotzdem bekannt sein um zu große Frustration bei dem Ersties zu vermeiden.

\subsection{Ämter}
Folgenden Unterstationen sind geplant:
\begin{itemize}
    \item Krankenkasse
    \item F-Vermittlung
    \item Auslandsamt
    \item Patentamt
    \item Prüfungsamt
    \item Meldeamt
\end{itemize}
\subsubsection{Krankenkasse}
Die Krankenkasse kommt in der \textbf{Questline Sport} als letzte Station vor. Sie soll Punkte für das Bonusprogramm gewähren, jedoch sollte sie sich dazu vorher noch vergewissern, dass die Punkte auch zu Recht vergeben werden. Beispiel Ideen dafür sind ein paar Hampelmänner, Kniebeugen oder Liegestütze.

\subsubsection{Job-Vermittlung}
\textbf{Questline Wirtschaftsjob 1.Station:} Der Erstie muss ein Formular abholen für die Firmengründung. Daraufhin kann er anfangen seine Mindblowing Idea auszuarbeiten, da es spontan entstand, natürlich auf Klopapier. Haben sie diese kommen sie wieder zurück zur Jobvermittlung.

\textbf{Questline Wirtschaftsjob 2.Station:} Die Firma muss verifiziert werden, und natürlich genügt sie nicht den strengen Standards, sie wurde auf Klopapier geschrieben, was erwartet man da auch. Kriterien die nicht genügend sein können sind zum Beispiel: Crafting, Modbar, Nachhaltigkeit, IoT oder auch Vegan, suchen sie sich was aus.

\textbf{Questline SHK 2.Station:} Der Antrag auf eine SHK stelle muss natürlich mehrfach und korrekt ausgefüllt werden, und das elektronische System ist natürlich noch nicht ausgereift, schließlich sind wir keine Informatiker, sondern Beamte. Das entsprechende Formular kann im Internet ausgedruckt werden. Die Aufgabe der Beamten ist zum Glück denkbar einfach, es soll nur überprüft werden ob alle Formulare auch korrekt ausgefüllt sind. Im Anschluss sollte der Erstie zum ZIH geschickt werden, denn schließlich benötigt jeder eine Imma Bescheinigung für eine SHK stelle.

\textbf{Questline SHK 3.Station:} Hat der Erstie erfolgreich seine Imma Bescheinigung geholt, so kann sie dort abgegeben werden und der Vertrag geht durch alle Instanzen der Uni, dem Land Sachsen und dem BND, das dauert auch nicht länger als 1 Semester (30min). Im Anschluss sollte der Vertrag aber auch Pünktlich abgeholt werden, Unpünktlichkeit darf auch gerne mit verlängerten Wartezeiten gestraft werden, zum Beispiel einem Ausgiebigen Gespräch mit dem Büro gegenüber, wenn der Erstie es nicht nötig hat sich zu beeilen, warum sollten sie es denn dann tun.

\subsubsection{Auslandsamt}
\textbf{Questline Wirtschaftsjob 4.Station:} Jeder gute Job wird im Ausland gemacht, schließlich gibt es dort auch viel mehr Leute als im eigenem Land, logisch oder? Ihre Aufgabe ist nun das Outsourcen, dauert zum Glück nur 10 bis 20 min.

\textbf{Questline Auslandssemester 2.Station:} Ersties wollen gerne ins Ausland, finden sie heraus welches das beste ist. Sie wissen nicht welches, müssen sie auch nicht, entscheide der Zufall. Eine Landkarte an der Tafel und ein paar Magneten sollten ihre Aufgabe lösen.


\subsubsection{Patentamt}
\textbf{Questline Wirtschaftsjob 3.Station:} Für eine erfolgreiche Firma benötigt man Investoren, haben die Jungründer 3 Investoren gefunden, so darf das Patentamt grünes Licht geben.

\textbf{Questline Auslandssemester 3.Station:} Ersties im Ausland müssen Projektarbeiten machen, und das Patentamt hat mehr als genug Arbeit. Ist das Projekt bearbeitet, so sollte eine kleine Präsentation am Flipchart nicht schwer fallen.

\subsubsection{Prüfungsamt}
\textbf{Questline Gremiensemester 4.Station:} Nach langer und harter Arbeit darf sich der Erstie ein Gremiensemester anrechnen lassen, natürlich muss das Prüfungsamt davon noch überzeugt werden, schließlich ist das zurücksetzen von Semestern keine Kleinigkeit. Ein wenig Engagement und Leidenschaft in einer Rede kann man da schon erwarten.

\textbf{Questline Regelstudienzeit 1/3.Station:} Kommt ein Erstie an und zeigt seine bestandenen Prüfung, so muss das Prüfungsamt tatsächlich einmal seiner Arbeit nachkommen, mein herzliches Beileid. Aber keine sorge, Stempel aufs Papier und fertig. Blöder weise kann das jedes Semster passieren.  

\textbf{Questline SHK 1.Station:} Ein Erstie möchte eine Stelle in der Uni habe, aber dazu benötigt er natürlich eine bestanden Prüfung in dem Fach, wo kämen wir denn hin wenn die Lehrenden keine Ahnung von dem Inhalt haben. Weisen sie einfach nach das das Modul bestanden ist und alle sind zufrieden.

\textbf{Quesline Auslandssemester 1.Station:} Sollten sie das Dokument von oben nach unten lesen, so sollten sie wissen was auch hier zu tun ist. Stellen sie einfach eine Bescheinigung über zwei bestandene Prüfungen aus und wünschen sie dem Erstie viel Spaß im Ausland.

\textbf{Questline durchschummeln 4.Station:} Tja Prüfung bescheinigen, sprich Stempeln und gut ist, auch wenn es komisch aussehen sollten, denken sie nicht weiter nach, ist nur mehr Arbeit für sie. Was soll schon passieren.

\subsubsection{Meldeamt}
\textbf{Questline Wohnungswechsel 3.Station:} Der Erstie muss seine Wohnung ordnungsgemäß anmelden, hat er das vor, so sollte es an dieser Station mit einem Stempel belohnt werden. Im Anschluss darf er nur nicht vergessen seine Adresse beim ZIH auch zu ändern, ein kleiner Hinweis wäre gut.
