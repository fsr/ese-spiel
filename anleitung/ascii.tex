\section{ascii}

\subsection{Allgemein}

Im ascii lernen die Erstis den Vereins- und Caféalltag kennen. Sie beschäftigen sich mit den Matesorten, der Vereinsarbeit, den Vernissages oder der Koorperation mit dem CD. 

\subsection{Aufbau}

Das ganze findet im ascii statt (Überraschung!). Man benötigt etwas Platz zum verheiraten und Sitzmöglichkeiten zum Schminken. Vor dem ascii steht ein Stehtisch auf dem man die Bilder mal kann.

Helfer: Mind. 3


\subsection{Materialien}
\begin{itemize}
    \item Schminkzeug
\item Hochzeitskram (Hüte, Schals, Kronen etc.)
\item Leere Kisten zum Stapeln
\item Bunte Stifte + Papier

\end{itemize}

Involvierte Questlines

Nachtwanderunng - Verkostung: 3 Matesorten werden verkostet und müssen bewertet werden. Falsche Bewertungen -> Abspülen, bei komplett falscher bewertung: Polizei gibt Hausverbot 


Große Liebe - Haikus schreiben: Eheversprechen schreiben bis das ascii zufrieden ist. (Zettel daneben legen was ein Haiku eigentlich ist)
- Hochzeit: Hut und Krone anbringen, Leute werden geschminkt und dann verheiratet

Gremiensemester - Vereinsarbeit
bspw: Fegen, Tassen spülen, Türme von Hanoi, Löffel/Tassen in der Fakultät finden

Regelstudienzeit - Praktikum in den Ferien
Bilder malen und präsentieren (wie Vernissage früher)
oder Vereinsarbeit s.o.

Partygänger - Wertmarke für Bier abholen
Tanzen um zu überzeugen 
Macarena


Allgelehrter - Matesorten auswendig lernen
Matefakten: Welche ist bio, welche Untersorten, Geschichte, Inhaltsstoffe etc.
5 Minuten zeit zum lernen, dann Test bestehen.

(Beispiel-) Fragen:
\begin{itemize}
    \item Welche Matesorte ist die älteste? \textrightarrow Club Mate (nicht geprüft)
    \item Welche Inhaltsstoffe sind in Club Mate, aber nicht in Kolle Mate? \textrightarrow 
    \item Welche Untersorten hat Club Mate? \textrightarrow Mate Cola, Wintermate, Granatmate, Icetee Mate, ..?
    \item Welche ist die blumigste Mate? \textrightarrow Flora Mate :)
\end{itemize}
