\documentclass[12pt,a4paper]{article}
\usepackage{german}

\begin{document}

\hbox to\textwidth{\hfill\small vertraulich}


\section*{Infos f"ur die FSI}

\subsection*{Protest}

Bei ein paar Ereignissen (neue H"ochststudiendauern o.\,"a.) ist 
es m"oglich, dagegen zu protestieren. Werden gen"ugend Leute 
mobilisiert, so k"onnen die Schweinereien abgewendet werden. 
Herrscht gen"ugend Chaos, ist die Demo also erfolgreich, so schickt 
bitte irgend jemanden zur Spielleitung, um den Erfolg zu verk"unden.

\subsection*{Laufbogen}

Die Spielleitung kontrolliert mit Gruppenlaufb"ogen, ob
negatvie Ereignisse tats"achlich umgesetzt werden. Am Anfang jeder
Runde informiert die Spielleitung die FSI, wieviele lebende Studis
eine Gruppe hat. Solltet ihr negative Punkte (Egoismus) vergeben,
teilt das bitte der Spielleitung mit. (Krank wird bei der FSI
hoffentlich niemand ;-)


\subsection*{Wahl}

Die Wahl findet zu einem bestimmten Zeitpunkt im Fachschaftszimmer statt,
Gruppen die w"ahlen gehen, bekommen 1 Informiertheitspunkt pro lebendem
Studi.

\subsection*{Output-Layout}

\begin{itemize}
\item
W"ahrend einer bestimmten Zeitspanne findet das Output-Layout statt.
Die Studies sind aufgefordert, einen Sch"uttelreim auszusagen. Wer 
dies kann (maximal einmal pro Gruppe), darf {\em w"urfeln\/} und bekommt 
die {\em Augenzahl als Informiertheitspunkte\/}, die dann auf die gesamte 
Gruppe verteilt werden k"onnen.
\item
Kommen viele Leute einer Gruppe auf einmal, gibt es zus"atzlich 
Gute-Freunde-Punkte. Wie viele, k"onnt ihr selbst entscheiden. 
Obergrenze ist die Anzahl der noch lebendes Studies in der 
Gruppe.
\end{itemize}

\subsection*{B"ucherb"orse}

Da unsere B"ucherb"orse "uberl"auft suchen wir Helfer, die die 
Anzahl der B"ucher z"ahlen. Wer richtig z"ahlt bekommt zwei 
Informiertheitspunkte. Kommen viele Leute einer Gruppe auf einmal, 
gibt es zus"atzlich Gute-Freunde-Punkte. Wie viele, k"onnt ihr selbst 
entscheiden. Obergrenze ist die Anzahl der noch lebendes 
Studis in der Gruppe.

\subsection*{Klausurensammlung}

Wer eine Klausurensammlung machen m"ochte, bekommt einen 
Informiertheitspunkt.

\subsection*{Sommerfest}

Wer beim Sommerfest mithilft bekommt Gute-Freunde- und Lebenserfahrungspunkte,
die anderen Gruppen bekommen Egoismuspunkte von der Spielleitung,
teilt der Spielleitung bitte mit, welche Gruppen das betrifft. 
Laut Ereignisinformation
verteilt die FSI ausserdem Gummib"archen, sollten wir solche
Tiere wg. Artenschutz nicht mehr halten, entf"allt das nat"urlich.

\subsection*{ESE-Wochenende}

Wer sich f"urs ESE-Wochenende anmeldet, bekommt Gute-Freunde- und 
Lebenserfahrungspunkte.

\end{document}
